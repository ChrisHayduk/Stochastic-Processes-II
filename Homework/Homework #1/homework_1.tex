\documentclass[12pt]{article}
 
\usepackage[margin=1in]{geometry}
\usepackage{amsmath,amsthm,amssymb, mathtools}
\usepackage[T1]{fontenc}
\usepackage{lmodern}
\usepackage{fixltx2e}
\usepackage[shortlabels]{enumitem}
\usepackage{mathrsfs}
\usepackage{kbordermatrix}

\usepackage{graphicx}
\usepackage{bbm}

\renewcommand{\kbldelim}{(}% Left delimiter
\renewcommand{\kbrdelim}{)}% Right delimiter
 
\newcommand{\N}{\mathbb{N}}
\newcommand{\R}{\mathbb{R}}
\newcommand{\Z}{\mathbb{Z}}
\newcommand{\Q}{\mathbb{Q}}
 
\newenvironment{theorem}[2][Theorem]{\begin{trivlist}
\item[\hskip \labelsep {\bfseries #1}\hskip \labelsep {\bfseries #2.}]}{\end{trivlist}}
\newenvironment{lemma}[2][Lemma]{\begin{trivlist}
\item[\hskip \labelsep {\bfseries #1}\hskip \labelsep {\bfseries #2.}]}{\end{trivlist}}
\newenvironment{exercise}[2][Exercise]{\begin{trivlist}
\item[\hskip \labelsep {\bfseries #1}\hskip \labelsep {\bfseries #2.}]}{\end{trivlist}}
\newenvironment{problem}[2][Problem]{\begin{trivlist}
\item[\hskip \labelsep {\bfseries #1}\hskip \labelsep {\bfseries #2.}]}{\end{trivlist}}
\newenvironment{question}[2][Question]{\begin{trivlist}
\item[\hskip \labelsep {\bfseries #1}\hskip \labelsep {\bfseries #2.}]}{\end{trivlist}}
\newenvironment{corollary}[2][Corollary]{\begin{trivlist}
\item[\hskip \labelsep {\bfseries #1}\hskip \labelsep {\bfseries #2.}]}{\end{trivlist}}
\newcommand{\textfrac}[2]{\dfrac{\text{#1}}{\text{#2}}}
\newcommand{\floor}[1]{\left\lfloor #1 \right\rfloor}

\newenvironment{amatrix}[1]{%
  \left(\begin{array}{@{}*{#1}{c}|c@{}}
}{%
  \end{array}\right)
}

\DeclareMathOperator*{\E}{\mathbb{E}}


\begin{document}

\title{Stochastic Processes II: Homework 1}

\author{Chris Hayduk}
\date{February 10, 2021}

\maketitle

\begin{problem}{I}
\end{problem}

We have that $M_n = X_n/(n+2)$. Since we start with $1$ red ball in the urn, we have that $X_0 = 1$. Note that $X_n \leq n+1$ for every $n$ because $X_n$ will increase by at most $1$ per ``turn'' but could stay the same, whereas $n$ will always increase by 1 unit per turn. Thus, we have that $0 < M_n = X_n/(n+2) \leq (n+1)/(n+2) < 1$ for all $n \in \mathbb{N}_0$. Hence, $E|M_n| < \infty$ for all $n$.\\

Now, since $M_n = X_n/(n+2)$, we have that $M_n$ is completely determined by $X_n$. Thus, this example satisfies the condition that $M_n$ be determined from the values for $X_n, \ldots, X_0$ and $M_0$.\\

Now consider,
\begin{align*}
M_{n+1} - M_n &= X_{n+1}/((n+1)+2) - X_n/(n+2)\\
&= X_{n+1}/(n+3) - X_n/(n+2)
\end{align*}

Then, since $X_n$ is given in $A_v = \{X_n = x_n, X_{n-1} = x_{n-1},\ldots, X_0 = x_0, M_0 = m_0\}$, we have
\begin{align*}
E\left[\frac{X_{n+1}}{n+3} - \frac{X_n}{n+2} \ | \ A_v\right] &= 1/(n+3)\cdot E[X_{n+1} \ | \ A_v] - 1/(n+2) \cdot E[X_n \ | \ A_v]\\
&= 1/(n+3)\cdot E[X_{n+1} \ | \ A_v] - X_n/(n+2)\\
&= 1/(n+3) \cdot \left(M_n \cdot (X_{n} + 1) + (1 - M_n) \cdot X_{n}\right) - X_n/(n+2)\\
&= 1/(n+3) \cdot \left(M_nX_{n} + M_n + X_{n} - M_nX_{n+1}\right) - X_n/(n+2)\\
&= 1/(n+3) \cdot \left(M_n + X_{n}\right) - X_n/(n+2)\\
&= 1/(n+3) \cdot \left(X_n/(n+2) + X_{n}\right) - X_n/(n+2)\\
&= \frac{X_n}{(n+2)(n+3)} + \frac{X_n}{n+3} - \frac{X_n}{n+2}\\
&= \frac{X_n}{(n+2)(n+3)} + \frac{X_n(n+2)}{(n+2)(n+3)} - \frac{X_n(n+3)}{(n+2)(n+3)}\\
&= \frac{X_n}{(n+2)(n+3)} - \frac{X_n}{(n+2)(n+3)}\\
&= 0
\end{align*}

as required. Hence, $(M_n)$ is a martingale with respect to $(X_n)$.

\begin{problem}{II}
\end{problem}

For each $n$, we have that $M_n = S_n^3 - 3nS_n = (X_1 + \cdots + X_n)^3 -  3n(X_1 + \cdots + X_n)$. Hence, $M_n$ is an algebraic expression with a finite sum of $1$s and $-1$s for each $n$, and thus $E|M_n| < \infty$. Furthermore, from the above, we see that $M_n$ can be written as a function of $X_1, \ldots, X_n$, so $M_n$ is fully determined by $M_0, X_0, \ldots, X_n$ as required.\\

Moving on to the increments, we have,
\begin{align*}
M_{n+1} - M_n &= ((S_{n+1})^3 - 3(n+1)(S_{n+1})) - (S_n^3 - 3nS_n)\\
&= 3nS_n - 3(n+1)S_{n+1} - S_n^3 + S_{n+1}^3
\end{align*}

This gives us,
\begin{align*}
E\left[3nS_n - 3(n+1)S_{n+1} - S_n^3 + S_{n+1}^3 \ | \ A_v\right] &= 3E\left[nS_n \ | \ A_v\right] - 3E\left[(n+1)S_{n+1} \ | \ A_v\right] -\\
&E\left[S_n^3 \ | \ A_v\right] + E\left[S_{n+1}^3 \ | \ A_v\right]\\
&= 3E\left[nS_n \ | \ A_v\right] - 3E\left[(n+1)(S_{n} + X_{n+1}) \ | \ A_v\right] \\
&- E\left[S_n^3 \ | \ A_v\right] + E\left[S_n^3 + 3S_n^2X_{n+1} + 3S_nX_{n+1}^2 + X_{n+1}^3 \ | \ A_v\right]\\
&= 3nS_n - 3(n+1)S_{n} - S_n^3 + S_{n}^3 + 3S_n^2 E\left[X_{n+1} \ | \ A_v \right]\\
&+ 3S_nE[X_{n+1}^2 \ | \ A_v] + E[X_{n+1}^3 \ | \ A_v]\\
&= -3S_n + 3S_n\\
&= 0
\end{align*}

Hence $(M_n)$ is a martingale with respect to $(X_n)$.

\begin{problem}{III}
\end{problem}

By Lemma 5.4 in Durrett and taking $A_v = \{X_n = x_n, X_{n-1} = x_{n-1}, \ldots, X_0 = x_0, M_0 = m_0\}$, we have,
\begin{align*}
\E\left[\Delta M_n \Delta M_m\right] &= \E\left[\E\left[\Delta M_n \Delta M_m \ | \ A_v\right]\right]\\
\end{align*}

Then, by Lemma 5.1 and property (iii) from the definition of martingales, we have,
\begin{align*}
\E\left[\E\left[\Delta M_n \Delta M_m \ | \ A_v\right]\right] &= \E\left[\Delta M_m\E\left[\Delta M_n  \ | \ A_v\right]\right]\\
&= \E\left[\Delta M_m \cdot 0 \right]\\
&= \E\left[0\right]\\
&= 0
\end{align*}

Now for the variance of $M_n$, we have
\begin{align*}
\E M_n^2 &= \E \left[\left((M_n - M_{n-1}) + (M_{n-1} - M_{n-2}) + \cdots + (M_1 - M_0) + M_0\right)^2\right]\\
&= \E [(M_n - M_{n-1})^2 + (M_{n-1} - M_{n-2})^2 + \cdots + (M_1 - M_0)^2 + M_0^2 + 2(M_n - M_{n-1})(M_{n-1} - M_{n-2})\\
& + \cdots + 2M_0(M_1 - M_0)]
\end{align*}

By the above arguments, all of the arguments where two intervals are multiplied together will have an expectation of $0$. Hence, we get,
\begin{align*}
\E M_n^2 = \sum_{i=0}^{n-1} \left[ \E (M_{i+1} - M_i) \right] + \E M_0^2
\end{align*}



\end{document}