\documentclass[12pt]{article}
 
\usepackage[margin=1in]{geometry}
\usepackage{amsmath,amsthm,amssymb, mathtools}
\usepackage[T1]{fontenc}
\usepackage{lmodern}
\usepackage{fixltx2e}
\usepackage[shortlabels]{enumitem}
\usepackage{mathrsfs}
\usepackage{kbordermatrix}

\usepackage{graphicx}
\usepackage{bbm}

\renewcommand{\kbldelim}{(}% Left delimiter
\renewcommand{\kbrdelim}{)}% Right delimiter
 
\newcommand{\N}{\mathbb{N}}
\newcommand{\R}{\mathbb{R}}
\newcommand{\Z}{\mathbb{Z}}
\newcommand{\Q}{\mathbb{Q}}
 
\newenvironment{theorem}[2][Theorem]{\begin{trivlist}
\item[\hskip \labelsep {\bfseries #1}\hskip \labelsep {\bfseries #2.}]}{\end{trivlist}}
\newenvironment{lemma}[2][Lemma]{\begin{trivlist}
\item[\hskip \labelsep {\bfseries #1}\hskip \labelsep {\bfseries #2.}]}{\end{trivlist}}
\newenvironment{exercise}[2][Exercise]{\begin{trivlist}
\item[\hskip \labelsep {\bfseries #1}\hskip \labelsep {\bfseries #2.}]}{\end{trivlist}}
\newenvironment{problem}[2][Problem]{\begin{trivlist}
\item[\hskip \labelsep {\bfseries #1}\hskip \labelsep {\bfseries #2.}]}{\end{trivlist}}
\newenvironment{question}[2][Question]{\begin{trivlist}
\item[\hskip \labelsep {\bfseries #1}\hskip \labelsep {\bfseries #2.}]}{\end{trivlist}}
\newenvironment{corollary}[2][Corollary]{\begin{trivlist}
\item[\hskip \labelsep {\bfseries #1}\hskip \labelsep {\bfseries #2.}]}{\end{trivlist}}
\newcommand{\textfrac}[2]{\dfrac{\text{#1}}{\text{#2}}}
\newcommand{\floor}[1]{\left\lfloor #1 \right\rfloor}

\newenvironment{amatrix}[1]{%
  \left(\begin{array}{@{}*{#1}{c}|c@{}}
}{%
  \end{array}\right)
}

\DeclareMathOperator*{\E}{\mathbb{E}}


\begin{document}

\title{Stochastic Processes II: Homework 3}

\author{Chris Hayduk}
\date{March 3, 2021}

\maketitle

\begin{problem}{I}
\end{problem}


\begin{problem}{II}
Durrett 5.9
\end{problem}

Let $X_1, X_2, \ldots$ be independent with $P(X_i = 1) = p$ and $P(X_i = -1) = q = 1-p$ where $p < 1/2$. Let $S_n = S_0 + \xi_1 + \cdots + \xi_n$ and let $V_0 = \min\{n \geq 0 : S_n = 0\}$. (5.13) implies $\E_xV_0 = x/(1-2p)$. If we let $Y_i = X - i - (p-q)$ and note that $EY_i = 0$ and $$\text{var}(Y_i) = \text{var}(X_i) = EX_iu^2 - (EX_i)^2$$ then it follows that $(S_n - (p-q)n)^2 - n(1-(p-q)^2)$ is a martingale. Moreover, when $S_0 = x$, the variance of $X_0$ is $$x \cdot \frac{1 - (p-q)^2}{(q-p)^3}$$
\begin{enumerate}[label=(\alph*)]

\item

\item

\item

\end{enumerate}


\begin{problem}{III}
LPW 4.2
\end{problem}

By Proposition 4.2, we have that,
\begin{align*}
||\mu P - vP||_{TV} &= \frac{1}{2} \sum_{x \in \mathcal{X}} |\mu P(x) - vP(x)|\\
&= \frac{1}{2} \sum_{x \in \mathcal{X}} \left| \sum_{y \in \mathcal{X}} \mu(y) P(y, x) - v(y)P(y, x) \right|\\
&= \frac{1}{2} \sum_{x \in \mathcal{X}} \left| \sum_{y \in \mathcal{X}} P(y,x)[\mu(y) - v(y)] \right|\\
&\leq \frac{1}{2} \sum_{x \in \mathcal{X}} \sum_{y \in \mathcal{X}} P(y,x)|\mu(y) - v(y)|\\
&= \frac{1}{2} \sum_{y \in \mathcal{X}} |\mu(y) - v(y)| \sum_{x \in \mathcal{X}} P(y,x)\\
&= \frac{1}{2} \sum_{y \in \mathcal{X}} |\mu(y) - v(y)| \cdot 1\\
&= \frac{1}{2} \sum_{y \in \mathcal{X}} |\mu(y) - v(y)|\\
&= ||\mu - v||_{TV}
\end{align*}

Hence, we have that $||\mu P - vP||_{TV} \leq ||\mu - v||_{TV}$ as required. In particular, we have that $||\mu P^{t+1} - \pi||_{TV} \leq ||\mu P^t - \pi||_{TV}$.\\

Now fix $t > 0$. By the above, we must have that,
\begin{align*}
&\max_{x \in \mathcal{X}} ||P^{t+1}(x, \cdot) - \pi||_{TV} \leq \max_{x \in \mathcal{X}} ||P^{t}(x, \cdot) - \pi||_{TV}\\
\iff &d(t+1) \leq d(t)
\end{align*}

and,
\begin{align*}
&\max_{x \in \mathcal{X}} ||P^{t+1}(x, \cdot) - P^{t+1}(y, \cdot)||_{TV} \leq \max_{x \in \mathcal{X}} ||P^{t}(x, \cdot) - P^{t}(y, \cdot)||_{TV}\\
\iff &\overline{d}(t+1) \leq \overline{d}(t)
\end{align*}


\begin{problem}{III}
LPW 4.4
\end{problem}


\end{document}