\documentclass[12pt]{article}
 
\usepackage[margin=1in]{geometry}
\usepackage{amsmath,amsthm,amssymb, mathtools}
\usepackage[T1]{fontenc}
\usepackage{lmodern}
\usepackage{fixltx2e}
\usepackage[shortlabels]{enumitem}
\usepackage{mathrsfs}
\usepackage{kbordermatrix}


\usepackage{graphicx}
\usepackage{bbm}

\DeclarePairedDelimiter{\ceil}{\lceil}{\rceil}

\renewcommand{\kbldelim}{(}% Left delimiter
\renewcommand{\kbrdelim}{)}% Right delimiter
 
\newcommand{\N}{\mathbb{N}}
\newcommand{\R}{\mathbb{R}}
\newcommand{\Z}{\mathbb{Z}}
\newcommand{\Q}{\mathbb{Q}}
 
\newenvironment{theorem}[2][Theorem]{\begin{trivlist}
\item[\hskip \labelsep {\bfseries #1}\hskip \labelsep {\bfseries #2.}]}{\end{trivlist}}
\newenvironment{lemma}[2][Lemma]{\begin{trivlist}
\item[\hskip \labelsep {\bfseries #1}\hskip \labelsep {\bfseries #2.}]}{\end{trivlist}}
\newenvironment{exercise}[2][Exercise]{\begin{trivlist}
\item[\hskip \labelsep {\bfseries #1}\hskip \labelsep {\bfseries #2.}]}{\end{trivlist}}
\newenvironment{problem}[2][Problem]{\begin{trivlist}
\item[\hskip \labelsep {\bfseries #1}\hskip \labelsep {\bfseries #2.}]}{\end{trivlist}}
\newenvironment{question}[2][Question]{\begin{trivlist}
\item[\hskip \labelsep {\bfseries #1}\hskip \labelsep {\bfseries #2.}]}{\end{trivlist}}
\newenvironment{corollary}[2][Corollary]{\begin{trivlist}
\item[\hskip \labelsep {\bfseries #1}\hskip \labelsep {\bfseries #2.}]}{\end{trivlist}}
\newcommand{\textfrac}[2]{\dfrac{\text{#1}}{\text{#2}}}
\newcommand{\floor}[1]{\left\lfloor #1 \right\rfloor}

\newenvironment{amatrix}[1]{%
  \left(\begin{array}{@{}*{#1}{c}|c@{}}
}{%
  \end{array}\right)
}

\DeclareMathOperator*{\E}{\mathbb{E}}


\begin{document}

\title{Stochastic Processes II: Homework 9}

\author{Chris Hayduk}
\date{May 14, 2021}

\maketitle

\begin{problem}{I}
LPW 12.3
\end{problem}

Let $P_L = (P + I)/2$ be the transition matrix of the lazy version of the chain with transition matrix $P$. Suppose $P_L$ has eigenvalue $\lambda < 0$ with corresponding eigenfunction $f$. Then,
\begin{align*}
P_L f &= (P/2 + I/2) f\\
&= (Pf + f)/2\\
&= \lambda f
\end{align*}

From the above equalities, we have,
\begin{align*}
&Pf/2 + f/2 = \lambda f\\
\iff &Pf/2 = \lambda f - f/2\\
\iff &Pf = 2 \lambda f - f\\
\iff &Pf = (2\lambda - 1)f
\end{align*}

Hence, $2\lambda - 1$ is an eigenvalue of $P$ and so, by Lemma 12.1(i), we have that,
\begin{align*}
&2 \lambda - 1 \geq -1\\
\iff &\lambda \geq 0
\end{align*}

We assumed that $\lambda < 0$, and hence we have a contradiction. Thus, there are no eigenvalues $\lambda$ of $P_L$ such that $\lambda < 0$, as required.

\newpage
\begin{problem}{II}
LPW 12.4
\end{problem}

We have that,
\begin{align*}
E_{\pi}(P^tf) &= \pi P^t f\\
&= \pi f\\
&= E_{\pi}(f)
\end{align*}

We have that the first eigenfunction $f_1 \equiv 1$, and so we have,
\begin{align*}
P^t f - E_{\pi}(P^t f) &= \sum_{j=1}^{| \mathcal{X} |} \langle f, f_j \rangle_{\pi} f_j \lambda_j^t\\
&= \sum_{j=2}^{| \mathcal{X} |} \langle f, f_j \rangle_{\pi} f_j \lambda_j^t
\end{align*}

Because the $f_j$'s are an orthonormal basis, this gives us,
\begin{align*}
\text{Var}_{\pi}(f) &= ||P^t f - E_{\pi}(P^tf)||^2_{\ell^2(\pi)}\\
&= \sum_{j=2}^{| \mathcal{X} |} \langle f, f_j \rangle_{\pi}^2 f_j \lambda_j^{2t}
\end{align*}

Now note that for all $\lambda_j$, we have that $1 - \gamma_* \geq \lambda_j$. Thus, we have,
\begin{align*}
\text{Var}_{\pi}(f) &= \sum_{j=2}^{| \mathcal{X} |} \langle f, f_j \rangle_{\pi}^2 f_j \lambda_j^{2t}\\
&\leq \sum_{j=2}^{| \mathcal{X} |} \langle f, f_j \rangle_{\pi}^2 f_j (1 - \gamma_*)^{2t}\\
&= (1 - \gamma_*)^{2t} \sum_{j=2}^{| \mathcal{X} |} \langle f, f_j \rangle_{\pi}^2 f_j
\end{align*} 

Moreover, we have that,
\begin{align*}
\sum_{j=2}^{| \mathcal{X} |} \langle f, f_j \rangle_{\pi}^2 &= E_{\pi}(f^2) - E^2_{\pi}(f)\\
&= \text{Var}_{\pi}(f)
\end{align*}

And so,
\begin{align*}
\text{Var}_{\pi}(f) &\leq (1 - \gamma_*)^{2t} \text{Var}_{\pi}(f)
\end{align*}

as required.

\newpage
\begin{problem}{III}
LPW 18.1
\end{problem}

Suppose that the $n$-th chain in a sequence of Markov chains satisfies,
\begin{align*}
\lim_{n \to \infty} d_n (c t^n_{\text{mix}}) - \begin{cases}
1 & \text{if } c < 1,\\
0 & \text{if } c > 1
\end{cases}
\end{align*}

Then for any $\gamma > 0$ and for $n$ large enough, we have,
\begin{align*}
t_{\text{mix}}(\epsilon) &\leq (1 + \gamma) t^n_{\text{mix}}\\
t_{\text{mix}}(1 - \epsilon) &\geq (1 - \gamma)t^n_{\text{mix}}
\end{align*}

These equations together yield,
\begin{align*}
\frac{t_{\text{mix}}(\epsilon)}{t_{\text{mix}}(1 - \epsilon)} \leq \frac{1 + \gamma}{1 - \gamma}
\end{align*}

If we let $\gamma$ approach $0$, then $n$ must approach $\infty$ and we get,
\begin{align*}
\lim_{n \to \infty} \frac{t_{\text{mix}}(\epsilon)}{t_{\text{mix}}(1 - \epsilon)} = 1
\end{align*}

as required.\\

Now suppose that we have,

\begin{align*}
\lim_{n \to \infty} \frac{t_{\text{mix}}(\epsilon)}{t_{\text{mix}}(1 - \epsilon)} = 1
\end{align*}

Fix $\gamma > 0$. Then for any $\epsilon > 0$ and for $n$ large enough, we have $t_{\text{mix}}(\epsilon) \leq (1 + \gamma)t^n_{\text{mix}}$. That is, $\lim_{n \to \infty} d_n((1 + \gamma)t^n_{\text{mix}}) \leq \epsilon$. Since this holds for all $\epsilon$,
\begin{align*}
\lim_{n \to \infty} d_n ((1 + \gamma)t^n_{\text{mix}}) = 0
\end{align*}

Moreover, $\lim_{n \to \infty} d_n ((1 - \gamma) t^n_{\text{mix}}) \geq 1 - \epsilon$ since $t_{\text{mix}}(1-\epsilon) \geq (1 - \gamma)t^n_{\text{mix}}$ for $n$ sufficiently large. Thus, we have that,
\begin{align*}
\lim_{n \to \infty} d_n ((1 - \gamma) t^n_{\text{mix}}) = 1
\end{align*}

\end{document}