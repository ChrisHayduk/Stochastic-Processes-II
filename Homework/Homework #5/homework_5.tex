\documentclass[12pt]{article}
 
\usepackage[margin=1in]{geometry}
\usepackage{amsmath,amsthm,amssymb, mathtools}
\usepackage[T1]{fontenc}
\usepackage{lmodern}
\usepackage{fixltx2e}
\usepackage[shortlabels]{enumitem}
\usepackage{mathrsfs}
\usepackage{kbordermatrix}


\usepackage{graphicx}
\usepackage{bbm}

\DeclarePairedDelimiter{\ceil}{\lceil}{\rceil}

\renewcommand{\kbldelim}{(}% Left delimiter
\renewcommand{\kbrdelim}{)}% Right delimiter
 
\newcommand{\N}{\mathbb{N}}
\newcommand{\R}{\mathbb{R}}
\newcommand{\Z}{\mathbb{Z}}
\newcommand{\Q}{\mathbb{Q}}
 
\newenvironment{theorem}[2][Theorem]{\begin{trivlist}
\item[\hskip \labelsep {\bfseries #1}\hskip \labelsep {\bfseries #2.}]}{\end{trivlist}}
\newenvironment{lemma}[2][Lemma]{\begin{trivlist}
\item[\hskip \labelsep {\bfseries #1}\hskip \labelsep {\bfseries #2.}]}{\end{trivlist}}
\newenvironment{exercise}[2][Exercise]{\begin{trivlist}
\item[\hskip \labelsep {\bfseries #1}\hskip \labelsep {\bfseries #2.}]}{\end{trivlist}}
\newenvironment{problem}[2][Problem]{\begin{trivlist}
\item[\hskip \labelsep {\bfseries #1}\hskip \labelsep {\bfseries #2.}]}{\end{trivlist}}
\newenvironment{question}[2][Question]{\begin{trivlist}
\item[\hskip \labelsep {\bfseries #1}\hskip \labelsep {\bfseries #2.}]}{\end{trivlist}}
\newenvironment{corollary}[2][Corollary]{\begin{trivlist}
\item[\hskip \labelsep {\bfseries #1}\hskip \labelsep {\bfseries #2.}]}{\end{trivlist}}
\newcommand{\textfrac}[2]{\dfrac{\text{#1}}{\text{#2}}}
\newcommand{\floor}[1]{\left\lfloor #1 \right\rfloor}

\newenvironment{amatrix}[1]{%
  \left(\begin{array}{@{}*{#1}{c}|c@{}}
}{%
  \end{array}\right)
}

\DeclareMathOperator*{\E}{\mathbb{E}}


\begin{document}

\title{Stochastic Processes II: Homework 5}

\author{Chris Hayduk}
\date{March 17, 2021}

\maketitle

\begin{problem}{I}
LPW 5.4
\end{problem}

\begin{enumerate}[label=\alph*)]

\item We have from the proof of Theorem 5.6 that for any coordinate $i$, the expected coupling time is,
\begin{align*}
E_{x, y}(\tau_i) \leq \frac{dn^2}{4}
\end{align*}

Thus, we have that, if $t = dn^2$, then,
\begin{align*}
P_{x, y}\{\tau_i > t\} &\leq \frac{E_{x, y}(\tau_i)}{t}\\
&\leq \frac{1}{t} \frac{dn^2}{4}\\
&= \frac{1}{4}
\end{align*}

Now suppose $t = 2dn^2$. By Remark 5.3, we know that this coupling is Markovian and hence, the probability that that $x$ and $y$ have not coupled between $t_0 = dn^2$ and $t = 2dn^2$ is still $1/4$. Thus, by induction, we have that if $t \geq kdn^2$,
\begin{align*}
P_{x, y}\{\tau_i > t\} &\leq P_{x, y}\{\tau_i > kdn^2\}\\
&\leq (1/4)^k
\end{align*}

\item We have that,
\begin{align*}
P\left\{\max_{1 \leq i \leq d} \tau_i > kdn^2\right\} \leq P\left(\sqcup_{i=1}^d \tau_i > kdn^2\right)
\end{align*}

Since the coordinates are independent of one another, our union in the second part of the above equation is disjoint. Hence, we have,
\begin{align*}
P\left(\sqcup_{i=1}^d \tau_i > kdn^2\right) &= \sum_{i=1}^d P\left(\tau_i > kdn^2\right)\\
&= \frac{d}{4^k}
\end{align*}

Now, if we take $k = \ceil[\bigg]{\log_4(d/\epsilon)}$, then we have,
\begin{align*}
P\left\{\max_{1 \leq i \leq d} \tau_i > kdn^2\right\} &\leq \frac{d}{4^k}\\
&= \frac{d}{4^{\ceil{\log_4(d/\epsilon)}}}\\
&= \frac{d}{\ceil{d/\epsilon}}\\
&\leq \frac{d}{d/\epsilon}\\
&= \epsilon
\end{align*}

And hence by Corollary 5.5, we have that,
\begin{align*}
d(t) &\leq \max_{x, y \in \mathcal{X}} P_{x, y}\{\tau_{\text{couple}} > t\}\\
&\leq \epsilon
\end{align*}

when $t > kdn^2$ and $k = \ceil[\bigg]{\log_4(d/\epsilon)}$. Hence,
\begin{align*}
t_{\text{mix}} &= kdn^2\\
&= \ceil[\bigg]{\log_4(d/\epsilon)}dn^2
\end{align*}

as required.
\end{enumerate}

\begin{problem}{II}
LPW 5.5
\end{problem}

Let us assume that we now have a finite $b-$ary tree with depth $k$. By the same coupling process as described in section 5.3.4, we have that,
\begin{align*}
\tau &= \min \{t \geq 0: X_t = Y_t \}\\
\tau_{\rho} &= \min\{t \geq 0: Y_t = \rho\}
\end{align*}

And $E(\tau) \leq E_{y_0}(\tau_{\rho})$. Moreover, the distance of $Y_t$ from the leaves is still a birth-and-death chain, this time with $p = \frac{1}{2} \cdot \frac{1}{b} = \frac{1}{2b}$ and $q = \frac{1}{2} - p = \frac{b-1}{2b}$. Then by (2.14), we have,
\begin{align*}
E_{y_0}(\tau_{\rho}) &\leq \frac{2b}{2-b} \left[k + \frac{b-1}{2b}\left(\frac{1 - (b-1)^k}{\frac{2-b}{2b}}\right)\right]\\
&=  \frac{2b}{2-b} \left[k + \frac{b-1}{2-b}\left(1 - (b-1)^k\right)\right]\\
&\leq -2b \left[k + b\left(1 - b^k\right)\right]\\
&= -2bk - 2b^2 + 2b^{k+2}\\
&= 2b(b^{k+1} - k - b)\\
&\leq 2bn
\end{align*}

And so,
\begin{align*}
t_{\text{mix}} &\leq 4(2bn)\\
&= 8bn
\end{align*}

\begin{problem}{III}
LPW 6.10
\end{problem}

Let $\tau_k$  be the first time that vertex $k$ has been reached.

\end{document}