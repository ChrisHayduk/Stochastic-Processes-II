\documentclass[12pt]{article}
 
\usepackage[margin=1in]{geometry}
\usepackage{amsmath,amsthm,amssymb, mathtools}
\usepackage[T1]{fontenc}
\usepackage{lmodern}
\usepackage{fixltx2e}
\usepackage[shortlabels]{enumitem}
\usepackage{mathrsfs}
\usepackage{kbordermatrix}


\usepackage{graphicx}
\usepackage{bbm}

\DeclarePairedDelimiter{\ceil}{\lceil}{\rceil}

\renewcommand{\kbldelim}{(}% Left delimiter
\renewcommand{\kbrdelim}{)}% Right delimiter
 
\newcommand{\N}{\mathbb{N}}
\newcommand{\R}{\mathbb{R}}
\newcommand{\Z}{\mathbb{Z}}
\newcommand{\Q}{\mathbb{Q}}
 
\newenvironment{theorem}[2][Theorem]{\begin{trivlist}
\item[\hskip \labelsep {\bfseries #1}\hskip \labelsep {\bfseries #2.}]}{\end{trivlist}}
\newenvironment{lemma}[2][Lemma]{\begin{trivlist}
\item[\hskip \labelsep {\bfseries #1}\hskip \labelsep {\bfseries #2.}]}{\end{trivlist}}
\newenvironment{exercise}[2][Exercise]{\begin{trivlist}
\item[\hskip \labelsep {\bfseries #1}\hskip \labelsep {\bfseries #2.}]}{\end{trivlist}}
\newenvironment{problem}[2][Problem]{\begin{trivlist}
\item[\hskip \labelsep {\bfseries #1}\hskip \labelsep {\bfseries #2.}]}{\end{trivlist}}
\newenvironment{question}[2][Question]{\begin{trivlist}
\item[\hskip \labelsep {\bfseries #1}\hskip \labelsep {\bfseries #2.}]}{\end{trivlist}}
\newenvironment{corollary}[2][Corollary]{\begin{trivlist}
\item[\hskip \labelsep {\bfseries #1}\hskip \labelsep {\bfseries #2.}]}{\end{trivlist}}
\newcommand{\textfrac}[2]{\dfrac{\text{#1}}{\text{#2}}}
\newcommand{\floor}[1]{\left\lfloor #1 \right\rfloor}

\newenvironment{amatrix}[1]{%
  \left(\begin{array}{@{}*{#1}{c}|c@{}}
}{%
  \end{array}\right)
}

\DeclareMathOperator*{\E}{\mathbb{E}}


\begin{document}

\title{Stochastic Processes II: Homework 6}

\author{Chris Hayduk}
\date{April 14, 2021}

\maketitle

\begin{problem}{I}
LPW 6.8
\end{problem}

If $A$ is the set of vertices in one of the complete graphs, observe that we have chosen $n$ of the $2n-1$ vertices of the glued graph. Hence $\pi(A) = n/(2n-1) \geq 1/2$.\\

For $x \not\in A$, we have,
\begin{align*}
P^t(x, A) &= 1 - \left(1 - \frac{1}{2n}[1 + o(1)]\right)^t
\end{align*}

We can use this to bound the total variation distance from below as follows,
\begin{align*}
||P^t(x, \cdot) - \pi||_{TV} &\geq \pi(A) - P^t(x, A)\\
&\geq (1 -  \frac{1}{2n}[1 + o(1)])^t - \frac{1}{2}
\end{align*}

We have that $$\log(1 - \frac{1}{2n}[1 + o(1)])^t - \frac{1}{2} \geq \frac{1}{4}$$ Now if $t < [4\frac{1}{2n}[1 + o(1)](1 - (\frac{1}{2n}[1 + o(1)])/2]^{-1}$, then
\begin{align*}
(1 - \frac{1}{2n}[1 + o(1)])^t - 1/2 \geq 1/4
\end{align*}

which gives us that $t_{mix}(1/4) \geq \frac{n}{2}[1 + o(1)]$


\begin{problem}{II}
LPW 6.9
\end{problem}

Observe that $\tau_{k+1} = T_{\tau_k} + 1$ and so $\tau_k = T_{\tau_{k-1}} + 1$. Hence,
\begin{align*}
m_k &= E(\tau_k)\\
&= E(T_{\tau_{k-1}} + 1)\\
&= E(T_{\tau_{k-1}}) + 1\\
\end{align*}

Similarly,
\begin{align*}
m_{k+1} &= E(T_{\tau_k}) + 1\\
&= E(T_{\tau_{k-1}} + t_k) + 1\\
&= E(T_{\tau_{k-1}}) + E(t_k) + 1\\
&= E(T_{\tau_k -1}) + 1 + E(t_k)\\
&= m_k + E(t_k)
\end{align*}

\begin{problem}{III}
LPW 7.2
\end{problem}

We have that,
\begin{align*}
Q(S, S^C) &= \sum_{x \in S, y \in S^C} Q(x,y)\\
&= \sum_{x \in S, y \in S^C} \pi(x) P(x,y)
\end{align*}

and,
\begin{align*}
Q(S^C, S) &= \sum_{x \in S^C,  y \in S} Q(x, y)\\
&= \sum_{x \in S, y \in S^C} \pi(x) P(x,y)
\end{align*}

Observe that, for $Q(S, S^C)$, we can split the summation into two sums which yields,
\begin{align*}
Q(S, S^C) &= \sum_{y \in S^C} \sum_{x \in C} \pi(x) P(x,y)
\end{align*}

Next, note that $\sum_{x \in C} \pi(x) P(x,y) = \sum_{x \in \mathcal{X}} \pi(x) P(x,y) - \sum_{x \in S^C} \pi(x) P(x,y)$. Making this substitution in the above gives us,
\begin{align*}
Q(S, S^C) &= \sum_{y \in S^C} [\sum_{x \in \mathcal{X}} \pi(x) P(x,y) - \sum_{x \in S^C} \pi(x) P(x,y)]\\
&= \sum_{y \in S^C} \sum_{x \in \mathcal{X}} \pi(x) P(x,y) - \sum_{x \in S^C} \pi(x) \sum_{y \in S^C} P(x,y)\\
&= \sum_{y \in S^C} \pi(y) - \sum_{x \in S^C} \pi(x) \left[ 1 - \sum_{y \in S} P(x, y) \right]\\
&= \sum_{y \in S^C} \pi(y) - \sum_{x \in S^C} \pi(x) + \sum_{x \in S^C} \sum_{y \in S} P(x, y) \pi(x) P(x, y)\\
&= \sum_{x \in S^C} \sum_{y \in S} P(x, y) \pi(x) P(x, y)
\end{align*}

But note that this is exactly the definition we gave for $Q(S^C, S)$. Hence, we must have that $Q(S, S^C) = Q(S^C, S)$.

\newpage
\begin{problem}{IV}
LPW 7.4
\end{problem}

\begin{enumerate}[label=(\alph*)]

\item Note that,
\begin{align*}
|\mathcal{X}| = (2^n - 1)(2^n - 2^1) \cdots (2^n - 2^{n-1})
\end{align*}

Hence, we have,
\begin{align*}
\lim_{n \to \infty} \frac{|\mathcal{X}|}{2^{n^2}} &= \lim_{n \to \infty} \frac{(2^n - 1)(2^n - 2^1) \cdots (2^n - 2^{n-1})}{2^{n^2}}\\
&= \frac{\lim_{n \to \infty} (2^n - 1)(2^n - 2^1) \cdots (2^n - 2^{n-1})}{\lim_{n \to \infty} 2^{n^2}}
\end{align*}



\item 

\end{enumerate}

\end{document}