\documentclass[12pt]{article}
 
\usepackage[margin=1in]{geometry}
\usepackage{amsmath,amsthm,amssymb, mathtools}
\usepackage[T1]{fontenc}
\usepackage{lmodern}
\usepackage{fixltx2e}
\usepackage[shortlabels]{enumitem}
\usepackage{mathrsfs}
\usepackage{kbordermatrix}

\usepackage{graphicx}
\usepackage{bbm}

\renewcommand{\kbldelim}{(}% Left delimiter
\renewcommand{\kbrdelim}{)}% Right delimiter
 
\newcommand{\N}{\mathbb{N}}
\newcommand{\R}{\mathbb{R}}
\newcommand{\Z}{\mathbb{Z}}
\newcommand{\Q}{\mathbb{Q}}
 
\newenvironment{theorem}[2][Theorem]{\begin{trivlist}
\item[\hskip \labelsep {\bfseries #1}\hskip \labelsep {\bfseries #2.}]}{\end{trivlist}}
\newenvironment{lemma}[2][Lemma]{\begin{trivlist}
\item[\hskip \labelsep {\bfseries #1}\hskip \labelsep {\bfseries #2.}]}{\end{trivlist}}
\newenvironment{exercise}[2][Exercise]{\begin{trivlist}
\item[\hskip \labelsep {\bfseries #1}\hskip \labelsep {\bfseries #2.}]}{\end{trivlist}}
\newenvironment{problem}[2][Problem]{\begin{trivlist}
\item[\hskip \labelsep {\bfseries #1}\hskip \labelsep {\bfseries #2.}]}{\end{trivlist}}
\newenvironment{question}[2][Question]{\begin{trivlist}
\item[\hskip \labelsep {\bfseries #1}\hskip \labelsep {\bfseries #2.}]}{\end{trivlist}}
\newenvironment{corollary}[2][Corollary]{\begin{trivlist}
\item[\hskip \labelsep {\bfseries #1}\hskip \labelsep {\bfseries #2.}]}{\end{trivlist}}
\newcommand{\textfrac}[2]{\dfrac{\text{#1}}{\text{#2}}}
\newcommand{\floor}[1]{\left\lfloor #1 \right\rfloor}

\newenvironment{amatrix}[1]{%
  \left(\begin{array}{@{}*{#1}{c}|c@{}}
}{%
  \end{array}\right)
}

\DeclareMathOperator*{\E}{\mathbb{E}}


\begin{document}

\title{Stochastic Processes II: Homework 4}

\author{Chris Hayduk}
\date{March 10, 2021}

\maketitle

\begin{problem}{I}
LPW 5.2
\end{problem}

Since $P\{\tau_{\text{couple}} \leq t_0\} \geq \alpha$, we have $$P\{\tau_{\text{couple}} > t_0\} \leq 1 - \alpha$$ Note that since this coupling is Markovian, the probability of not coupling on any length $t_0$ is the same. Hence $$P\{\tau_{\text{couple}} > kt_0\} \leq (1 - \alpha)^k$$

\begin{problem}{II}
\end{problem}

Observe that for $(X_n)$, we have,
\begin{align*}
S_X &= \{g \in G \ : \ \mu(g) > 0 \}\\
&= \{3\}
\end{align*}

This set generates the following group:
\begin{align*}
\langle 3, 1 \rangle 
\end{align*}

Hence, by Prop 2.13, we have that $(X_n)$ is not irreducible. On the other hand, we have
\begin{align*}
S_Y &= \{g \in G \ : \ \mu(g) > 0 \}\\
&= \{3, 5\}
\end{align*}

and so the set $S_Y$ generates the following group:
\begin{align*}
\langle 1, 3, 5, 7 \rangle 
\end{align*}

Thus, $(Y_t)$ is in fact irreducible. Note that in $G$, every element is its own inverse. That is $g = g^{-1}$ for all $g \in G$. Hence we have, 
\begin{align*}
\mu(g) = \mu(g^{-1})
\end{align*}

and
\begin{align*}
\nu(g) = \nu(g^{-1})
\end{align*}

Thus, both $\mu$ and $\nu$ are symmetric.

\begin{problem}{III}
\end{problem}

We have $X \sim \text{Poisson}(a)$, $Z \sim \text{Poisson}(b-a)$, and $Y = X + Z$. Hence, $Y \sim \text{Poisson}(b)$. We have that $(X, Y)$ is a coupling of $\mu$ and $\nu$. Observe that $X = Y$ iff $Z = 0$. Thus, $P\{X \neq Y\} = P\{Z \neq 0\}$. We have,
\begin{align*}
|| \mu - \nu ||_{TV} &\leq \inf \{P\{X \neq Y\} : \ (X, Y) \text{ is a coupling of } \mu \text{ and } \nu\}\\
&\leq P\{X \neq Y\}\\
&= P\{Z \neq 0\}\\
&= \frac{(b-a)^0 e^{-0}}{0!}\\
&= 1
\end{align*}

\end{document}