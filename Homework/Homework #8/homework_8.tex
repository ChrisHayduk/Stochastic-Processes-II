\documentclass[12pt]{article}
 
\usepackage[margin=1in]{geometry}
\usepackage{amsmath,amsthm,amssymb, mathtools}
\usepackage[T1]{fontenc}
\usepackage{lmodern}
\usepackage{fixltx2e}
\usepackage[shortlabels]{enumitem}
\usepackage{mathrsfs}
\usepackage{kbordermatrix}


\usepackage{graphicx}
\usepackage{bbm}

\DeclarePairedDelimiter{\ceil}{\lceil}{\rceil}

\renewcommand{\kbldelim}{(}% Left delimiter
\renewcommand{\kbrdelim}{)}% Right delimiter
 
\newcommand{\N}{\mathbb{N}}
\newcommand{\R}{\mathbb{R}}
\newcommand{\Z}{\mathbb{Z}}
\newcommand{\Q}{\mathbb{Q}}
 
\newenvironment{theorem}[2][Theorem]{\begin{trivlist}
\item[\hskip \labelsep {\bfseries #1}\hskip \labelsep {\bfseries #2.}]}{\end{trivlist}}
\newenvironment{lemma}[2][Lemma]{\begin{trivlist}
\item[\hskip \labelsep {\bfseries #1}\hskip \labelsep {\bfseries #2.}]}{\end{trivlist}}
\newenvironment{exercise}[2][Exercise]{\begin{trivlist}
\item[\hskip \labelsep {\bfseries #1}\hskip \labelsep {\bfseries #2.}]}{\end{trivlist}}
\newenvironment{problem}[2][Problem]{\begin{trivlist}
\item[\hskip \labelsep {\bfseries #1}\hskip \labelsep {\bfseries #2.}]}{\end{trivlist}}
\newenvironment{question}[2][Question]{\begin{trivlist}
\item[\hskip \labelsep {\bfseries #1}\hskip \labelsep {\bfseries #2.}]}{\end{trivlist}}
\newenvironment{corollary}[2][Corollary]{\begin{trivlist}
\item[\hskip \labelsep {\bfseries #1}\hskip \labelsep {\bfseries #2.}]}{\end{trivlist}}
\newcommand{\textfrac}[2]{\dfrac{\text{#1}}{\text{#2}}}
\newcommand{\floor}[1]{\left\lfloor #1 \right\rfloor}

\newenvironment{amatrix}[1]{%
  \left(\begin{array}{@{}*{#1}{c}|c@{}}
}{%
  \end{array}\right)
}

\DeclareMathOperator*{\E}{\mathbb{E}}


\begin{document}

\title{Stochastic Processes II: Homework 8}

\author{Chris Hayduk}
\date{April 28, 2021}

\maketitle

\begin{problem}{I}
LPW 8.9
\end{problem}

Let us assume we are working with a $3$ card deck. We will consider the distribution when $T = 3$.

\begin{enumerate}[\alph*)]

\item Since both cards are marked on every transposition and each selection is uniform and independent, we have that each card has a $2/3$ chance of being marked. At $T = 3$, there are $2$ cards already marked and $1$ card that is unmarked.

\item Since the right-hand card is marked on every transposition and each right-hand card is chosen uniformly, we have that each card has a $1/3$ chance of being marked. At $T = 3$, there are $2$ cards already marked and $1$ card that is unmarked. Hence, we have a $2/3$ chance of marking a marked card again, and a $1/3$ chance of marking the unmarked card.

\end{enumerate}

\begin{problem}{II}
LPW 12.1
\end{problem}

\begin{enumerate}[\alph*)]

\item By the hint, we will let $||f||_{\infty} = \max_{x \in \mathcal{X}} |f(x)|$. We have that,
\begin{align*}
||Pf||_{\infty} &= \max_{x \in \mathcal{X}} |P(x,y) f(y)|
\end{align*}

Since $0 \leq P(x,y) \leq 1$ for all $x,y \in \mathcal{X}$, we have that $|P(x,y) f(y)| < |f(y)|$ for all $x,y$. Hence,
\begin{align*}
||Pf||_{\infty} &= \max_{x \in \mathcal{X}} |P(x,y) f(y)|\\
&\leq \max_{x \in \mathcal{X}} |f(x)|\\
&= ||f||_{\infty}
\end{align*}

Now suppose $f$ is an eigenfunction with corresponding eigenvalue $\lambda$. The,
\begin{align*}
||Pf||_{\infty} &= ||\lambda f||_{\infty}\\
&= \max_{x \in \mathcal{X}} | \lambda f(x)|\\
&= |\lambda| \max_{x \in \mathcal{X}}  |f(x)|\\
&= |\lambda| ||f||_{\infty}
\end{align*}

From the first part of our proof, we have that,
\begin{align*}
||Pf||_{\infty} &= |\lambda| ||f||_{\infty}\\
&\leq ||f||_{\infty}
\end{align*}

This final inequality implies that $|\lambda| \leq 1$.

\item Assume that $\mathcal{T}(x) \subset 2 \mathbb{Z}$. Then every time $t$ such that $P^t(x, x) > 0$ is a multiple of $2$.

\item

\end{enumerate}

\begin{problem}{III}
LPW 12.2
\end{problem}

Let $P$ be irreducible and let $A$ be a matrix with $0 \leq A(i, j) \leq P(i, j)$ and $A \neq P$. Since $A \neq P$, we must have $A(i, j) < P(i,j)$ for some $i,j$. By 12.1(a), we have that every eigenvalue $\lambda$ of $P$ satisfies $|\lambda| \leq 1$. Now suppose $f$ is an eigenfunction of $A$ with eigenvalue $\lambda_1$ of $A$ and $\lambda_2$ of $P$. We have,
\begin{align*}
||Af||_{\infty} &= |\lambda_1| ||f||_{\infty}\\
&< ||Pf||_{\infty}\\
&= |\lambda_2| ||f||_{\infty}
\end{align*}

Dividing through by $||f||_{\infty}$ yields $|\lambda_1| < |\lambda_2| \leq 1$, and so $|\lambda_1| < 1$.
\end{document}