\documentclass[12pt]{article}
 
\usepackage[margin=1in]{geometry}
\usepackage{amsmath,amsthm,amssymb, mathtools}
\usepackage[T1]{fontenc}
\usepackage{lmodern}
\usepackage{fixltx2e}
\usepackage[shortlabels]{enumitem}
\usepackage{mathrsfs}
\usepackage{kbordermatrix}


\usepackage{graphicx}
\usepackage{bbm}

\DeclarePairedDelimiter{\ceil}{\lceil}{\rceil}

\renewcommand{\kbldelim}{(}% Left delimiter
\renewcommand{\kbrdelim}{)}% Right delimiter
 
\newcommand{\N}{\mathbb{N}}
\newcommand{\R}{\mathbb{R}}
\newcommand{\Z}{\mathbb{Z}}
\newcommand{\Q}{\mathbb{Q}}
 
\newenvironment{theorem}[2][Theorem]{\begin{trivlist}
\item[\hskip \labelsep {\bfseries #1}\hskip \labelsep {\bfseries #2.}]}{\end{trivlist}}
\newenvironment{lemma}[2][Lemma]{\begin{trivlist}
\item[\hskip \labelsep {\bfseries #1}\hskip \labelsep {\bfseries #2.}]}{\end{trivlist}}
\newenvironment{exercise}[2][Exercise]{\begin{trivlist}
\item[\hskip \labelsep {\bfseries #1}\hskip \labelsep {\bfseries #2.}]}{\end{trivlist}}
\newenvironment{problem}[2][Problem]{\begin{trivlist}
\item[\hskip \labelsep {\bfseries #1}\hskip \labelsep {\bfseries #2.}]}{\end{trivlist}}
\newenvironment{question}[2][Question]{\begin{trivlist}
\item[\hskip \labelsep {\bfseries #1}\hskip \labelsep {\bfseries #2.}]}{\end{trivlist}}
\newenvironment{corollary}[2][Corollary]{\begin{trivlist}
\item[\hskip \labelsep {\bfseries #1}\hskip \labelsep {\bfseries #2.}]}{\end{trivlist}}
\newcommand{\textfrac}[2]{\dfrac{\text{#1}}{\text{#2}}}
\newcommand{\floor}[1]{\left\lfloor #1 \right\rfloor}

\newenvironment{amatrix}[1]{%
  \left(\begin{array}{@{}*{#1}{c}|c@{}}
}{%
  \end{array}\right)
}

\DeclareMathOperator*{\E}{\mathbb{E}}


\begin{document}

\title{Stochastic Processes II: Homework 8}

\author{Chris Hayduk}
\date{April 28, 2021}

\maketitle

\textit{Note:} Worked with Raghu and George.

\begin{problem}{I}
LPW 8.9
\end{problem}

Let us assume we are working with a $3$ card deck. We will consider the distribution when $T = 3$.

\begin{enumerate}[\alph*)]

\item Suppose we have two of the three cards checked. Without loss of generality, assume card $3$ is unchecked in the third position. We only check the third card if it is part of a transposition. Hence, the only possible transpositions we can introduce where we mark the third card is by transposing it with one of the first two cards. That is, if $A_i$ is the i-th position in the deck, we only mark card $3$ if it ends up in $A_1$ or $A_2$ (that is, it is transposed with $1$ or with $2$). Hence, the resulting permutation given that two cards are already checked only has positive probability if $3$ is in one of the first two positions. The other $2$ possible permutations of $3$ cards (where card $3$ is in position $3$) have probability $0$. Hence, this is not uniform.

\item Suppose we have two of the three cards checked. Without loss of generality, assume card $3$ is unchecked and in the third position. We only check the third card if it is the right-hand card of a transposition. Hence, in order to get a permutation where $3$ is in the third position, we would need to swap it with card $1$ or card $2$ as the left-hand size of a transposition and then choose $3$ as the right-hand side of a transposition that sends it to $A_3$. Hence, it requires at least $2$ steps from our initial set up for $3$ to be marked in position $A_3$. However, we can get $3$ marked in position $A_1$ or $A_2$ in only $1$ step
by swapping $3$ as the right-hand side of a transposition with one of the first two cards. Hence, since it requires more steps to get $3$ marked in the final position, the probability of that permutation must be lower than the probability of a permutation where $3$ ends up in $A_1$ or $A_2$. Thus, this is not uniform.
\end{enumerate}

\newpage
\begin{problem}{II}
LPW 12.1
\end{problem}

\begin{enumerate}[\alph*)]

\item By the hint, we will let $||f||_{\infty} = \max_{x \in \mathcal{X}} |f(x)|$. We have that,
\begin{align*}
||Pf||_{\infty} &= \max_{x \in \mathcal{X}} |P(x,y) f(y)|
\end{align*}

Since $0 \leq P(x,y) \leq 1$ for all $x,y \in \mathcal{X}$, we have that $|P(x,y) f(y)| < |f(y)|$ for all $x,y$. Hence,
\begin{align*}
||Pf||_{\infty} &= \max_{x \in \mathcal{X}} |P(x,y) f(y)|\\
&\leq \max_{x \in \mathcal{X}} |f(x)|\\
&= ||f||_{\infty}
\end{align*}

Now suppose $f$ is an eigenfunction with corresponding eigenvalue $\lambda$. The,
\begin{align*}
||Pf||_{\infty} &= ||\lambda f||_{\infty}\\
&= \max_{x \in \mathcal{X}} | \lambda f(x)|\\
&= |\lambda| \max_{x \in \mathcal{X}}  |f(x)|\\
&= |\lambda| ||f||_{\infty}
\end{align*}

From the first part of our proof, we have that,
\begin{align*}
||Pf||_{\infty} &= |\lambda| ||f||_{\infty}\\
&\leq ||f||_{\infty}
\end{align*}

This final inequality implies that $|\lambda| \leq 1$.

\item Assume that $\mathcal{T}(x) \subset 2 \mathbb{Z}$. $-1$ is a $2$-nd root of unity because $-1^2 = 1$. Let $\mathcal{C}_j = \{x \in \mathcal{X}: P^{2m + j}(x_0, x) > 0 \text{ for some } m\}$ for $j = 0, 1, 2$. We have that there is a unique $j(x) \in \{0, 1\}$ such that $x \in \mathcal{C}_{j(x)}$ and  if $P(x,y) > 0$ then $j(y) = j(x) + 1 \mod 2$.\\

Let $f: \mathcal{X} \to \mathbb{C}$ be defined by $f(x) = -1^{j(x)}$. We have that, for some $\ell \in \mathbb{Z}$,
\begin{align*}
Pf(x) = \sum_{y \in \mathcal{X}} P(x,y) (-1)^{j(y)} &= -1^{j(x) + 1 \mod 2}\\
&= (-1)^{j(x) + 1 + 2\ell} = -1 (-1)^{j(x)} = -1 f(x) 
\end{align*}

Thus, $f(x)$ is an eigenfunction of $P$ with eigenvalue $-1$.\\

Now let $-1$ be an eigenvalue of $P$. Choose $x$ such that $|f(x)| = r := \max_{y \in \mathcal{X}} |f(y)|$. Since
\begin{align*}
-1f(x) = Pf(x) = \sum_{y \in \mathcal{X}} P(x,y)f(y)
\end{align*}

taking absolute values shows that 
\begin{align*}
r \leq \sum_{y \in X} P(x,y) |f(y)| \leq r
\end{align*}

We conclude that if $P(x, y) > 0$, then $|f(y)| = r$. By irreducibility, $|f(y)| = r$ for all $y \in \mathcal{X}$.\\

Since the average of complex numbers of norm $r$ has norm $r$ if and only if all the values have the same angle, it follows that $f(y)$ has the same value for all $y$ with $P(x,y) > 0$. Therefore, if $P(x,y) > 0$, then $f(y) = -1f(x)$. Now fix $x_0 \in \mathcal{X}$ and define for $j = 0, 1$,
\begin{align*}
\mathcal{C}_j = \{z \in \mathcal{X}: f(z) = -1^j f(x_0)\}
\end{align*}

It is clear that if $P(x,y)>0$ and $x \in \mathcal{C}_j$, then $x \in \mathcal{C}_{j + 1 \mod 2}$. It is clear that if $t \in \mathcal{T}(x_0)$, then $2$ divides $t$ and hence $\mathcal{T}(x_0) \subset 2\mathbb{Z}$, as required.

\item We now generalize the previous proof. Assume that $a$ divides $\mathcal{T}(x)$. If $b$ is the gcd of $\mathcal{T}(x)$, then $a$ divides $b$. If $\omega$ is an $a$-th root of unity, then $\omega^b = 1$. Let $\mathcal{C}_j = \{x \in \mathcal{X}: P^{mb + j}(x_0, x) > 0 \text{ for some } m\}$ for $j = 0, \ldots, b$. We have that there is a unique $j(x) \in \{0, \ldots, b-1\}$ such that $x \in \mathcal{C}_{j(x)}$ and  if $P(x,y) > 0$ then $j(y) = j(x) + 1 \mod b$.\\

Let $f: \mathcal{X} \to \mathbb{C}$ be defined by $f(x) = \omega^{j(x)}$. We have that, for some $\ell \in \mathbb{Z}$,
\begin{align*}
Pf(x) = \sum_{y \in \mathcal{X}} P(x,y) \omega^{j(y)} &= \omega^{j(x) + 1 \mod b}\\
&= \omega^{j(x) + 1 + b\ell} = \omega \omega^{j(x)} = \omega f(x) 
\end{align*}

Thus, $f(x)$ is an eigenfunction of $P$ with eigenvalue $\omega$.\\

Now let $\omega$ be an $a$-th root of unity and suppose that $\omega f = Pf$ for some $f$. Choose $x$ such that $|f(x)| = r := \max_{y \in \mathcal{X}} |f(y)|$. Since
\begin{align*}
\omega f(x) = Pf(x) = \sum_{y \in \mathcal{X}} P(x,y)f(y)
\end{align*}

taking absolute values shows that 
\begin{align*}
r \leq \sum_{y \in X} P(x,y) |f(y)| \leq r
\end{align*}

We conclude that if $P(x, y) > 0$, then $|f(y)| = r$. By irreducibility, $|f(y)| = r$ for all $y \in \mathcal{X}$.\\

Since the average of complex numbers of norm $r$ has norm $r$ if and only if all the values have the same angle, it follows that $f(y)$ has the same value for all $y$ with $P(x,y) > 0$. Therefore, if $P(x,y) > 0$, then $f(y) = \omega f(x)$. Now fix $x_0 \in \mathcal{X}$ and define for $j = 0, \ldots, k-1$,
\begin{align*}
\mathcal{C}_j = \{z \in \mathcal{X}: f(z) = \omega^j f(x_0)\}
\end{align*}

It is clear that if $P(x,y)>0$ and $x \in \mathcal{C}_j$, then $x \in \mathcal{C}_{j + 1 \mod k}$. It is clear that if $t \in \mathcal{T}(x_0)$, then $k$ divides $t$ and hence $\mathcal{T}(x_0) \subset k\mathbb{Z}$, as required.

\end{enumerate}

\begin{problem}{III}
LPW 12.2
\end{problem}

Let $P$ be irreducible and let $A$ be a matrix with $0 \leq A(i, j) \leq P(i, j)$ and $A \neq P$. Since $A \neq P$, we must have $A(i, j) < P(i,j)$ for some $i,j$. By 12.1(a), we have that $||Pf||_{\infty} \leq ||f||_{\infty}$ where $||f||_{\infty} = \max_{x \in \mathcal{X}} |f(x)|$. Let $f'(x_0)$ be the largest eigenfunction of $P$ and let us define $f$ and $\lambda$ to be an eigenfunction and its corresponding eigenvalue of $A$, respectively. Define $||f||_{\infty} = |f(x_1)|$ so that $|f(y)| \leq |f(x_1)|$ for all $y \in \mathcal{X}$. Thus,
\begin{align*}
||Af||_{\infty} = ||\lambda f||_{\infty} &= \max_{x \in \mathcal{X}} \left | \sum_{y \in \mathcal{X}} A(x,y) f(y) \right |\\
&\leq \max_{x \in \mathcal{X}} \left| \sum_{y \in \mathcal{X}} A(x, y) f(x_1) \right|\\
&= |f(x_1)| \max_{x \in \mathcal{X}} \sum_{y \in \mathcal{X}} A(x, y)\\
&< |f(x_1)| \max_{x \in \mathcal{X}} \sum_{y \in \mathcal{X}} P(x, y)\\
&= \frac{|f(x_1)|}{|f'(x_0)|} \max_{x \in \mathcal{X}} \sum_{y \in \mathcal{X}} P(x, y) \cdot |f'(x_0)|\\
&= \frac{|f(x_1)|}{|f'(x_0)|} \max_{x \in \mathcal{X}} \sum_{y \in \mathcal{X}} \left| P(x, y) \cdot f'(x_0) \right|\\
&= \frac{|f(x_1)|}{|f'(x_0)|} ||f'||_{\infty}\\
&= \frac{|f(x_1)|}{|f'(x_0)|} |f'(x_0)|
\end{align*}

Since $||\lambda f||_{\infty} = \lambda |f(x_1)| < \frac{|f(x_1)|}{|f'(x_0)|} |f'(x_0)| = |f(x_1)|$, then $\lambda < 1$ as desired.

\end{document}